\chapter{7.\hspace{0.5em}Woche}\label{wo7}

\section{Prototype}\label{wo7_1}

\begin{enumerate}[label={\Roman*)}]
	\item Am Freitag pr\"asentierten Andi und ich unseren bisherigen Stand der Dinge, dies beinhaltete die Registrierung mit
vorhandener Nutzerverwaltung und ein bis dato ausgereiftes Datenbankmodell.
	\item Weil meine zugewiesene Aufgabe nicht, wie geplant, fertig wurde, beriet ich mich mit meinem Team und suchte nach
m\"oglichen L\"osungen. Die von uns pr\"aferierte Variante war, das ich auf dem von Andi bereitgestellten Server entwickel und
somit die Infrastruktur des Servers nutzen kann, auf dem das Versenden von E-Mails problemlos erfolgen kann nach geringf\"ugiger
Konfiguration der Configdatei, auch ohne Authentifizierung, welche lokal ben\"otigt wird.
	\item Nachdem ich mir von Andi die ben\"otigten Zugangsdaten f\"ur eine ssh-Verbindung besorgt habe, fing ich an dort die
Funktionalit\"at der E-Mailversendung zu testen und nach erfolgreichem Test, mit der eigentlichen Implementierung der
E-Mailfunktionalit\"at f\"ur unsere bestehende Anwendung zu beginnen.
	\item Zum Ende der Woche stand dann eine grundlegende Funktion zum Versenden von E-Mails, diese war zwar nicht sehr
ausgereift aber sie lieferte alle geforderten Funktionen. Dies zu bewerkstelligen bedurfte bei mir die gesamte Zeit die ich diese
Woche f\"ur das Projekt eingeplant hatte.
	\item Zum Abschlu\ss{} dieser Woche haben wir uns in der Universit\"at nach Teams separiert getroffen um den jeweiligen
Fortschritt zu besprechen und um weitere Verbesserungsvorschl\"age zu erhalten und das weitere Vorgehen zu besprechen.
	\item Weiterhin haben wir von den Dozenten das Burn-Down-Chart erkl\"art bekommen, um so unseren aktuellen Stand der
Dinge festhalten zu k\"onnen und jederzeit bereit zu sein unseren Stand nach au\ss{}en hin korrespondieren zu k\"onnen.
\end{enumerate}