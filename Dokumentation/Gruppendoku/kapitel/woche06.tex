\chapter{6.\hspace{0.5em}Woche}\label{wo6}

\section{Prototype}\label{wo6_1}

\begin{enumerate}[label={\Roman*)}]
	\item Zur Einarbeitung in Lift habe ich mehrere Tutorien durchgearbeitet, um so einen kleinen Eindruck \"uber die
M\"oglichkeiten zu erhalten. So habe ich zum Beispiel einen Chat nach Anleitung implementiert, dies ging erstaunlich schnell,
jedoch zeigte dieses Beispiel schon eine gro\ss{}e Schwachstelle dieses Frameworks bez\"uglich unseres Projektes, da sehr wenig
Code ben\"otigt wurde um mehr oder weniger komplexe Funktionen zu implementieren. An sich ist dies eine sehr angenehme und
praktische Tatsache, jedoch kenne ich aus Erfahrung, dass solch ein Framework sehr viel Zeit f\"ur die Einarbeitung ben\"otigt,
leider steht uns im Rahmen des Projektes nicht genug Zeit daf\"ur zur Verf\"ugung.
	\item Am Wochenanfang haben wir dann nocheinmal im Team besprochen wie wir weiter verfahren, da jeder die Erfahrung
gemacht hat, das Lift zwar sehr wenig Code ben\"otigt um komplexe Funktionen zu bieten, aber keiner von uns wirklich verstanden
hatte wie dies in der Tat realisiert wird vom Framework. Aufgrund dieser Tatsache entschieden wir uns den Framework zu wechseln
und haben uns dann in Play 2.0 eingearbeitet.
	\item Auch hierf\"ur richteten wir dann die ben\"otigten Abh\"angigkeiten ein und begannen mit der Implementierung einer
grundlegenden Webseite. Leider kostete uns diese Einsicht sehr viel Zeit, sodass wir nicht die von uns geplanten Aufgaben
umsetzen konnten und im Plan deutlich hinterher h\"angen.
	\item Meine eigentliche Aufgabe dieser Iteration war das Versenden von E-Mails zur Verf\"ugung zu stellen, damit wir
zumindest einen Teil der Benachrichtigungen m\"oglichst fr\"uh nutzen k\"onnen.
	\begin{enumerate}[label={\arabic*}]
		\item Zum Versenden von E-Mails habe ich mich eingelesen, wie unter Linux E-Mails versendet werden k\"onnen.
		\item Nachdem ich lernte, dass diverse Programme zur Verf\"ugung stehen um dies zu bewerkstelligen, entschied ich
mich f\"ur postfix, da dies am einfachsten einzurichten sein sollte und auch neuer ist als die anderen Programme.
		\item Ich bewerkstelligte, von meinem Rechner lokal E-Mails versenden zu k\"onnen, diese landeten aufgrund
fehlender Authentifizierung und Verschl\"u\ss{}elung jedoch im Spam-Ordner der getesteten E-Mailanbieter(Gmail und GMX).
		\item Wenn unsere Website E-Mails versendet, sollten diese jedoch nicht im Spam-Ordner des jeweiligen
Empf\"angers landen, deswegen versuchte ich mittels tls die Authentifizierung zu erm\"oglichen und alle ben\"otigten Zertifikate
bereit zu stellen.
		\item Leider war dies nicht so einfach wie gedacht und gelesen, was auch daran lag, das viele Einstellungen in
der Configdatei, je nach gelesener Quelle, anders eingestellt werden sollten. Es war also leider nicht eindeutig ermittelbar
welche Einstellungen daf\"ur genau ben\"otigt sind.
		\item Diese Probleme endeten mit dem Ergebnis, dass ich in dieser Woche zwar viel Neues dazu gelernt habe, jedoch
kein vorzeigbares Ergebnis bieten konnte. Allerdings habe ich mich auch schon damit besch\"aftigt, welche Plugins oder andere
M\"oglichkeiten es gibt, um mit Scala E-Mails zu versenden, es fehlten also \glqq nur noch\glqq die Grundlagen um es lokal testen
zu k\"onnen.
	\end{enumerate}
\end{enumerate}