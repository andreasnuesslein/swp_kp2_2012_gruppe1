\chapter{3.\hspace{0.5em}Woche}\label{wo3}

\section{Ideate}\label{wo3_1}

\begin{enumerate}[label={\Roman*)}]
	\item Digitalisierung der erarbeiteten L\"osungsideen des Brainstormings f\"ur
	\item \"Uberlegung zur Realisierbarkeit einzelner L\"osungen und Teill\"osungen
	\item Treffen mit dem Team zur Abstimmung der bevorzugten L\"osungen/Teill\"osungen welche verfolgt werden sollen
	\item Erneute Erarbeitung von m\"oglichen Teill\"osungen und L\"osungen im Team
	\item Ermittlung der zu verfolgenden Ans\"atze
	\item Digitalisierung der Ergebnisse aus dem Teamtreffen
	\item Entwurf der Pr\"asentation
	\begin{enumerate}[label={\arabic*}]
		\item Das Vorstellen der Pr\"asentation am Freitagstreffen lief ohne gro\ss{}e Probleme ab, vorgetragen haben
Basti und Maiky, die beide sehr \"uberzeugt haben und unsere Ideen hervorragend r\"uber bringen konnten.
		\item Besonders erfreulich war die Reihenfolge der Vortr\"age aus unserer Sicht, da bei den drei Gruppen vor uns
schon in den Diskussionen signalisiert wurde das unsere beiden Ideen anscheinend einen Nerv getroffen haben, was m\"ogliche
Tester/Nutzer impliziert und somit eindeutig positiv zu bewerten war.
		\item Einziger Nachteil an diesem Treffen war, die, aus meiner Sicht ungerechtfertigte, Kritik bez\"uglich einer
Kleiderordnung. Auch wenn ich das Prinzip nachvollziehen kann und mich dem unterwerfen w\"urde sofern mein Arbeitgeber dies
forderte, so repr\"asentiere ich als Student auch eine offene, aufgeschlo\ss{}ene Haltung die, selbstverst\"andlich bei Wahrung
eines gewissen optischen Niveaus, nicht im geringsten etwas \"uber mein fachliches K\"onnen aussagt. Allerdings bin ich, meinem
Team zu liebe, das sich sehr f\"ur dieses Projekt engagiert und f\"ur das ich eine gewisse Mitverantwortung trage, bereit f\"ur
die Abschlu\ss{}pr\"asentation einen Kompromiss einzugehen.
	\end{enumerate}
\end{enumerate}