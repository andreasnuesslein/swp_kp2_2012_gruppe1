\chapter{9.\hspace{0.5em}Woche}\label{wo9}

\section{Prototype}\label{wo9_1}

\begin{enumerate}[label={\Roman*)}]
	\item Ich habe die Ermittlung der Favoriten aus den bestehenden Touren implementiert und mich damit besch\"aftigt welche
M\"oglichkeiten es zum Caching gibt. Letztendlich habe ich mich f\"ur die von Play 2.0 mit EHCache bereitgestellte
Standardversion entschieden und dieses implementiert. Dabei ist allerdings noch ausstehend, inwiefern der Cache funktioniert, da
dies nur im Feldtest wirklich ersichtlich wird.
	\item Ich habe mir Gedanken zur Neuverteilung der Aufgaben, welche Tu zugewiesen waren, gemacht, da dieser das Projekt
kurzfristig und unerwartet verlassen hat. Dies wirft unser gesamtes Team leider sehr zur\"uck, da die Aufgabe der automatischen
Bestellung und das Kl\"aren mit MyTaxi ob dies in Wolfsburg verf\"ugbar sein wird zugewiesen war.
	\item Weiterhin habe ich die initialen Templates f\"ur Touren aus der Vorgabe der FE-Taxiliste in die Datenbank
\"ubernommen und diese werden nun zusammen mit den Favoriten ermittelt, dabei werdern zuerst die Favoriten bestimmt und wenn
diese nicht ausreichend sind, werden weitere Templates aus der FE-Taxiliste zugef\"ugt bis die gew\"unschte Anzahl(8 Templates)
erreicht ist und somit eine gen\"ugend gro\ss{}e Auswahl an sofort w\"ahlbaren Templates bereit steht. Sollten jedoch noch keine
Favoriten bestehen, so werden nur die Templates aus der FE-Taxiliste angezeigt.
\end{enumerate}