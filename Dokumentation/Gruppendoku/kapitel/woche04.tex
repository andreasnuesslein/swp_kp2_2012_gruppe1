\chapter{4.\hspace{0.5em}Woche}\label{wo4}

\section{Paper-Prototype}\label{wo4_1}

\begin{enumerate}[label={\Roman*}]
	\item Erarbeitung des Paper-Prototypes f\"ur die Planung von Taxifahrten mit dem Schwerpunkt Wolfsburg.
	\item Im Teamtreffen trugen wir die Ideen zusammen und koordinierten die Vorteile unserer Einzell\"osungen.
	\begin{enumerate}[label={\arabic*}]
		\item Dabei war ein Hauptproblem die Koordinierung der Gruppe, das Verstehen von einzelnen Vor- und Nachteilen
der jeweiligen Ideen, weshalb ich vermehrt versucht habe anderen im Team zu erkl\"aren was an welcher
Idee gut ist und so versucht habe z\"ugiger voran zu kommen. 
		\item W\"ahrend dem Teamtreffen haben wir uns auf eine finale L\"osung geeinigt, welche wir grob ausarbeiteten.
Die Nachbearbeitung erfolgte Zuhause, in digitaler Form, erneut, jedoch von einem anderen Teammitglied.
		\item Nach der Ausarbeitung des Prototypen haben wir uns einen Kommilitonen, welcher am Projekt selbst
unbeteiligt ist, als Tester gesucht und mit diesem dann den Prototypen getestet und auch erste Schwachstellen ermittelt.
		\item Diese Schwachstellen waren zun\"achst \"ubersichtlicher Natur, weswegen wir sie nachbearbeiteten.
	\end{enumerate}
	\item Im Treffen am Freitag haben wir dann den finalen Prototypen getestet, an 3 verschiedenen Nutzern, und teilweise
unterschiedliches Feedback erhalten, aber teilweise auch gleiches Feedback.
	\begin{enumerate}[label={\arabic*}]
		\item Das Hauptproblem bei unserer Ausarbeitung war die Namensgebung, wir nutzten Aktive Reisen f\"ur die selbst
geplanten Reisen, dies verwirrte alle Nutzer, da es den Schlu\ss{} zulies dort bereits aktive Reisen anderer zu sehen, was
allerdings nicht geplant war.
		\item Wir testeten an 2 Nutzern zuerst die Webseitenversion unseres Prototypen und danach die Appversion, bei
diesen beiden Tests gab es reichlich Feedback zur Webseite aber nur geringf\"ugig zur App, da das Konzept bereits bekannt war
und, laut Aussagen der Tester, sogar einfacher und sch\"oner war.
		\item Am dritten und letzten Tester tauschten wir die Reihenfolge und erhielten unerwartet viel Feedback zur
Gestaltung unserer Elemente in der Appversion, da diese zuvor gut ankam und als einfach und einleuchtend beschrieben wurde, somit
haben wir zu beiden Versionen schlie\ss{}lich einiges an Feedback erhalten was wir im Nachhinein verarbeiten konnten.
		\item Alles in Allem war das Feedback jedoch nur geringf\"ugig, es waren also keinerlei gr\"o\ss{}ere
\"Anderungen notwendig die ein erneutes Testen der angepassten Version veranlassen w\"urden.
	\end{enumerate}
\end{enumerate}