\chapter{5\hspace{0.5em}Woche}\label{wo5}

\section{Prototype}\label{wo5_1}

\begin{enumerate}[label={\Roman*)}]
	\item Da die Abgabe der Wochenaufgabe diesmal sehr kurzfristig statt finden sollte, haben wir uns direkt nach dem Treffen
am Freitag in ein nahegelegenes TU-Geb\"aude begeben und dort unsere User Stories ausgearbeitet und priorisiert.
	\item Nach der Priorisierung erfolgte noch die Aufteilung in einzelne Aufgaben, die den jeweilgien User Stories
zugeh\"orig sind.
	\item Weil wir die Wochenaufgabe bereits am Freitag der Vorwoche erledigt hatten, blieb mehr Zeit f\"ur die eigentlichen
Aufgaben, welche mit Beginn der praktischen Arbeit anfielen. Daher haben wir uns dazu entschlo\ss{}en, dass sich jeder Gedanken
macht, welche Programmiersprache und welchen Framework wir nutzen wollen. Bei unserem n\"achsten Teamtreffen am Dienstag
beschlossen wir dann uns an Scala zu versuchen und einen geeigneten Framework zu ermitteln.
	\item F\"ur das Kennen lernen des verwendeten Frameworks und um gemeinsam z\"ugig starten zu
k\"onnen, haben wir uns am Donnerstag noch einmal getroffen und dort die Einarbeitung in verschiedene Frameworks probiert und uns
im Laufe des Treffens auf Lift geeinigt, da die Einarbeitung dort z\"ugig brauchbare Ergebnisse lieferte und zu Beginn einfach
erschien.
	\item Um die Motivation des Teams weiterhin so hoch zu halten, bug ich, auf Wunsch mehrerer Teammitglieder, einen
Kuchen(folgend gezeigt). Dies wurde dann ein veganer Zupfkuchen, da ich dem Team somit nicht nur einen leckeren Kuchen bieten
konnte, sondern ihnen auch zeigen konnte, das die vegane Lebensweise mehr zu bieten hat als man zu Beginn denkt. Das Team nahm den
Kuchen sehr positiv auf und es hat die Arbeitsatmosph\"are merklich aufgelockert.
\begin{center}
	\includegraphics[width=17cm]{img/2012-05-17_10-31-00.jpg}
\end{center}
	\item Nachdem wir die Nutzung von Scala mit Lift beschlo\ss{}en, vereinbarten wir, uns \"uber das Wochenende in den
Framework einzuarbeiten und nachzuvollziehen, wie die Registrierung und Speicherung der angelegten User von statten geht um so
das Grundger\"ust unserer Webseite z\"ugig erarbeiten zu k\"onnen.
	\item Damit dies bei jedem auf gleichem Stand beginnen kann, haben wir einander geholfen die daf\"ur ben\"otigten
Programme und Abh\"angigkeiten einzurichten.
\end{enumerate}