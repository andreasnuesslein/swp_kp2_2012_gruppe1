%webservice
\subsection{Prolog}\label{ssec:WebProlog}

As we begun with the implementation of the beforehand planned paper-prototype, we researched what technology and which
framework we should use.
After some thoughts about which programming language we should choose, we did decide to use Scala, as it offers a wide range of
interesting functionality, like functional programming or use of plain Java-code.
With the programming language chosen, a big step had been done, but another big one was not yet done, as we did not know which
framework we should use with Scala. As every member of the team did not have any experience in working with Scala, we did some
research and tried what we thought could be a possible option. That way we could gain a little insight into the framework and how
it works and what it has to offer. At this point in time we tried Play2.0(\url{http://www.playframework.org/}) and
Lift(\url{http://liftweb.net/}). Due to its simplicity in programming and the efficiency of the written code, we decided to use
Lift.\newline
With these decisions done, the next step to go was to create a basic website and some basic functionality, as planned for the
first iteration, in which we did plan on having some time to get friendly with a new framework and a new language.
Half a week later, with a very busy weekend for every team member, we evaluated if Lift was indeed a good choice and if we are
confident to be able to create a product as planned in the given time slot.
Unfortunately we all had the same thoughts about Lift and came to the solution that it would be better if we would use Play2.0 as
its use is not as complicated as it is with Lift. Lift is very efficient when it comes to written code and its results, but to
understand what is going on in the framework, so that we could create a good product the way we want it to be, would have needed
more time for familiarizing. But time was not that abundant, as we had a deadline every two weeks until which we had to have a
presentable result.
For that reason we did fall back onto Play2.0 and had to invest more time as planned for the first iteration as we were out of
plan already without even having anything to use.

\subsection{Description}\label{ssec:WebDesc}

The TaWusel webservice consists only of a few pages, which are shown in the appendix and described as follows.
During this project we did hit a deadline every two weeks, which led to a differing prototype every two weeks due to feedback at
any presentation and testing. With this feedback we were able to crystallize the needs and wishes of the clients and therefore
optimizing the product along with the clients wishes.\newline

When you use our product the first time, you will see the welcome page(figure \ref{img:WebLogin}), which contains some brief
features our product has to offer and of course you are able to login or register.\newline


Once you are registered or have logged in, you will see our main page(figure \ref{img:WebMain}). Our main page is one big list
separated into three parts. Each part of the list is marked by an underlying color that specifies the nature of the shown entry.
\indent Green rows are the entries which have either been created by you, or which you have joined as other users did create these
entries. If you are the one that created the our you will notice this as your name in the passengers column will be bold, as one
person is needed for calling a taxi and obviously the most qualified for that task is the one who created the tour.\newline
\indent Rows that are light blue are your personal favorites, which means, you have chosen these, at most 5 routes, more than any
other route and therefore these will be shown for you, so that your are able to plan a tour with the same specifics with just
two clicks.
If you just registered and did not yet plan any tour, you will have five entries as well, but these 5 entries are not yet your
personal favorites, they are just five beforehand chosen routes derived from the "FE-Taxilist", as these routes are said to
be often used.\newline
\indent And the last part of the main page are the white underlain rows. These are tours created by other users which you can
join if you wish to do so.\newline

If you now wish to create a new tour or to create one of your favorites as a new tour, you simply click the button at the end of
the favorite-row or above the table and then a modal panel will pop up(figure \ref{img:WebCreate}). This panel allows you to
choose the starting destination and the ending destination as well as the time the tour should start or end. These options are
available for creating a new tour as well as for creating a favorite tour, which means you have to do two clicks for creating a
favorite but if you want to take the same route as your favorite but another time then all you have to do is to change the time
or date and this will save you time, as the destinations are already pre-chosen.\newline

Up until now the main functionality has been described, and even though this is the most important part, we still have more to
offer.
Of course, if you are registered at our product you may want to change some of your given data at a specific point in time, for
example when your "BIK" changes, your email-address or your phone number. To do so we offer you a profile page where that is
possible. And due to as much simplicity as possible for the user we decided to use that same page for every data that is related
to your account.\newline
That said, what other data might be of interest for you as a user?\newline
\indent For one, you should be able to decide on your own for which purpose you want our service to contact you and remind you of
a planned tour or for someone joining or leaving your tour and of course in which manner this should happen. If you wish to be
informed via email than you may choose this option and may leave the SMS-option blank or you may want to be contacted via SMS to
your given mobile phone number.\newline
\indent Another information that could come in very handy is a history that is been kept track of(figure \ref{img:WebHistory}).
This history is keeping track of every tour you took part in and shows you every data for this tour, from the strating and ending
destinations and the time and date over the status of that tour(if it was booked, cancelled or pending as it should have begun)
and even who else took part in this tour.
With these informations you can easily look up when did you take a taxi, where did you take a taxi and who else did take it with
you. Very convenient if you do your invoice, especially as this prototype is extendable to send an email, containing all relevant
data for your invoice, with just one click or to print that data or whatever you need.
\indent As the figure \ref{img:WebHistory} shows, there is one other tab on the profile page, the favorites tab. There you can
reset your favorites which accumulated over time. It helps you to always have personal favorites. For example if you have to take
one tour for half a year three times a week, this tour has been done a lot of times and therefore would probably be your first
favorite, if no other has been taken more often. But after that time you may never have to use that tour again, probably due to
finishing the project, and now you have one of your five favorite slots filled with this one tour you may never have to take
again, then it is probably a good choice to reset this specific favorite.\newline
\indent These features describe our whole product and its features, and of course you are also able to logout.

\subsection{Architecture}\label{ssec:WebArchitecture}


\subsection{Installation guide - Webservice}\label{ssec:WebInst}

The rough install instructions can also be found in the \texttt{INSTALL.md} file.
\begin{itemize}

\item As TaWusel uses the Play!2 framework (Scala version) obviously the first requirement is said framework. To obtain it, the
best way is to follow the official installation documentation from the Play!
website\footnote{\url{http://www.playframework.org/}}.

\item The next step is to set up a MySQL-database\footnote{In theory, other SQL-DBMS should work as well, since TaWusel uses
JDBC for its connections. However we have only used MySQL so far.} for your project.
As installing MySQL can obviously not be part of this documentation, we shall only briefly point towards the official
website\footnote{\url{http://dev.mysql.com/doc/refman/5.6/en/installing.html}} and name
XAMPP\footnote{\url{http://www.apachefriends.org/en/xampp.html}} and MAMP\footnote{\url{http://www.mamp.info}} as two easy
alternatives
for installing MySQL on Windows and Mac OS X respectively.\\
Example code for setting up the database:
\begin{verbatim}
CREATE DATABASE tawusel;
CREATE USER tawusel;
GRANT ALL ON tawusel.* TO tawusel@localhost IDENTIFIED BY 'pass';
\end{verbatim}
\small{To use other credentials, the \texttt{conf/application.conf} has to be changed accordingly.}

\item Lastly go to \url{https://github.com/nutztherookie/tawusel} and download TaWusel. Unzip, go to the project's root folder and
run \texttt{play run}.\\
Point your web browser to \url{http://localhost:9000/}


\end{itemize}