%prototype

The following chapter comprises the main activity of the project: the implementation of the TaWusel webservice and the TaWusel
Android application. At the beginning we take a look at the basic architecture design of the whole service in section
\ref{sec:GArc}. Afterwards the database design is described in a brief section.

\emptyRow
Subsequently to the fundamentals the section \ref{sec:Webs} takes a closer look at the implementation of the web service.

\emptyRow
Last but not least the focus of section \ref{sec:Andr} will be on the Android application. Here you will find a brief description
of its functions (see \ref{ssec:AndrDes}), some design thoughts regarding the class diagram and a description of the classes which
are used by the app (see \ref{ssec:AndrArc}) as well as an installation guide to run the app on your mobile device (see
\ref{ssec:AndrInst}). 

\emptyRow
You can retrieve more information if you follow the link to our issue tracker
(\url{https://github.com/nutztherookie/swp_kp2_2012_gruppe1/issues}) Here most of our task which came up while the implementation
are covered. There are also some issues which are labeled with the prefix "wishlist", these are suggestions how one could improve
the service in the future. The whole document with all estimated users stories can be found in appendix \ref{chp:US}.

\clearpage

\section{General architecture}\label{sec:GArc}
%general architecture
\begin{figure}[ht]
	\centering
	\includegraphics[width=0.8\textwidth]{images/Architekturdiagramm}
	\caption{architecture diagram}
	\label{img:Arch}
\end{figure}
The web service has a direct connection with the database and consists of several greater parts itself. Those are the pictured
folders. The same goes for the Android application, but the Android application does use the web service for several purposes,
like consistent data storage.



\clearpage
\section{Database modeling}
%db modeling
\begin{figure}[ht]
	\centering
	\includegraphics[width=0.8\textwidth]{images/Datenbank_Model}
	\caption{database model}
	\label{img:DBMod}
\end{figure}

\clearpage
\section{Webservice}\label{sec:Webs}
%webservice
\subsection{Description}\label{ssec:WedDesc}

\includegraphics[width=16cm]{images/TaWusel_Web_Login.png}\label{img:WebLogin}
blabla


\includegraphics[width=16cm]{images/TaWusel_Web_main.png}\label{img:WebLogin}



\begin{figure}[h]
	\includegraphics[width=16cm]{images/TaWusel_Web_create.png}
	\caption{modal panel for creating a tour}
	\label{img:WebLogin}
\end{figure}



\begin{figure}[h]
	\includegraphics[width=16cm]{images/TaWusel_Web_history.png}
	\caption{profile page - history tab}
	\label{img:WebLogin}
\end{figure}
\subsection{Installation guide - Webservice}\label{ssec:WebInst}

The rough install instructions can also be found in the \texttt{INSTALL.md} file.
\begin{itemize}

\item As TaWusel uses the Play!2 framework (Scala version) obviously the first requirement is said framework. To obtain it, the
best way is to follow the official installation documentation from the Play!
website\footnote{\url{http://www.playframework.org/}}.

\item The next step is to set up a MySQL-database\footnote{In theory, other SQL-DBMS should work as well, since TaWusel uses
JDBC for its connections. However we have only used MySQL so far.} for your project.
As installing MySQL can obviously not be part of this documentation, we shall only briefly point towards the official
website\footnote{\url{http://dev.mysql.com/doc/refman/5.6/en/installing.html}} and name
XAMPP\footnote{\url{http://www.apachefriends.org/en/xampp.html}} and MAMP\footnote{\url{http://www.mamp.info}} as two easy
alternatives
for installing MySQL on Windows and Mac OS X respectively.\\
Example code for setting up the database:
\begin{verbatim}
CREATE DATABASE tawusel;
CREATE USER tawusel;
GRANT ALL ON tawusel.* TO tawusel@localhost IDENTIFIED BY 'pass';
\end{verbatim}
\small{To use other credentials, the \texttt{conf/application.conf} has to be changed accordingly.}

\item Lastly go to \url{https://github.com/nutztherookie/tawusel} and download TaWusel. Unzip, go to the project's root folder and
run \texttt{play run}.\\
Point your web browser to \url{http://localhost:9000/}


\end{itemize}

\clearpage
\section{Android App}\label{sec:Andr}
%android
\subsection{Installation guide}\label{ssec:AndrInst}

\textbf{1.} Make sure you locate the tawusel.properties into the assets folder. Here you have to replace the json server
property ("http://10.0.2.2:9000/") by the address of the server your tawusel service is running on.\\
\textbf{2.} Generate your tawusel.apk.\\
\textbf{3.} Install the apk on your mobile phone.\\
(remark: the current version works stable on android 2.3.3 but its working on higher versions is not guaranteed)

